%%%%%%%%%%%%%%%%%%%%%%%%%%%%%%%%%%%%%%%%%
% University Assignment Title Page 
% LaTeX Template
% Version 1.0 (27/12/12)
%
% This template has been downloaded from:
% http://www.LaTeXTemplates.com
%
% Original author:
% WikiBooks (http://en.wikibooks.org/wiki/LaTeX/Title_Creation)
%
% License:
% CC BY-NC-SA 3.0 (http://creativecommons.org/licenses/by-nc-sa/3.0/)
% 
% Instructions for using this template:
% This title page is capable of being compiled as is. This is not useful for 
% including it in another document. To do this, you have two options: 
%
% 1) Copy/paste everything between \begin{document} and \end{document} 
% starting at \begin{titlepage} and paste this into another LaTeX file where you 
% want your title page.
% OR
% 2) Remove everything outside the \begin{titlepage} and \end{titlepage} and 
% move this file to the same directory as the LaTeX file you wish to add it to. 
% Then add \input{./title_page_1.tex} to your LaTeX file where you want your
% title page.
%
%%%%%%%%%%%%%%%%%%%%%%%%%%%%%%%%%%%%%%%%%
%\title{Title page with logo}
%----------------------------------------------------------------------------------------
%	PACKAGES AND OTHER DOCUMENT CONFIGURATIONS
%----------------------------------------------------------------------------------------

\documentclass[12pt]{article}
\usepackage[english]{babel}
\usepackage[utf8x]{inputenc}
\usepackage{amsmath}
\usepackage{graphicx}
\usepackage[colorinlistoftodos]{todonotes}
%\newdateformat{mydate}{\monthname[\THEMONTH] \THEYEAR} 
\begin{document}

\begin{titlepage}

\newcommand{\HRule}{\rule{\linewidth}{0.5mm}} % Defines a new command for the horizontal lines, change thickness here

\center % Center everything on the page
 
%----------------------------------------------------------------------------------------
%	HEADING SECTIONS
%----------------------------------------------------------------------------------------

\textsc{\LARGE Scuola Superiore Sant'Anna and University of Pisa}\\[1.0cm] % Name of your university/college
\textsc{\Large Embedded System Design}\\[0.5cm] % Major heading such as course name
\textsc{\large Technical Report}\\[0.5cm] % Minor heading such as course title

%----------------------------------------------------------------------------------------
%	TITLE SECTION
%----------------------------------------------------------------------------------------

\HRule \\[0.4cm]
{ \huge \bfseries Wav Audio Playback}\\[0.4cm] % Title of your document
\HRule \\[1.5cm]
 
%----------------------------------------------------------------------------------------
%	AUTHOR SECTION
%----------------------------------------------------------------------------------------

\begin{minipage}{0.5\textwidth}
\begin{flushleft} \large
\emph{Author:}\\
Diego Alejandro Parra Guzman % Your name

\emph{Author:}\\
Yaneakay  % Your name
\end{flushleft}
\end{minipage}
~
\begin{minipage}{0.4\textwidth}
\begin{flushright} \large
\emph{Supervisor:} \\
Dr. Antonio Di Natale % Supervisor's Name
\end{flushright}
\end{minipage}\\[2cm]

% If you don't want a supervisor, uncomment the two lines below and remove the section above
%\Large \emph{Author:}\\
%John \textsc{Smith}\\[3cm] % Your name


%----------------------------------------------------------------------------------------
%	LOGO SECTION
%----------------------------------------------------------------------------------------

\includegraphics[width=0.5\textwidth]{SantanaPisa.png}\\[1cm] % Include a department/university logo - this will require the graphicx package
 
%----------------------------------------------------------------------------------------


%----------------------------------------------------------------------------------------
%	DATE SECTION
%----------------------------------------------------------------------------------------

%\mydate\today
{\large \today}
%{\large \today}\\[2cm] % Date, change the \today to a set date if you want to be precise

\vfill % Fill the rest of the page with whitespace

\end{titlepage}


\section{abstract}
On this document is contained the principal topics developed during the module embedded system design. where is exploited in detail some that is called model base design; Therefore the document is focused on three developing phases such as: requirements, behavior, architecture, and verification.
\newline
\newline
The above schema will be applied in one specific project called audio playback  which consist of a simple graphical application able to reproduce files in format WAV allocated in an external flash disk commonly called SDCard.    


\section{Introduction}

%Your introduction goes here! Some examples of commonly used commands and features are listed below, to help you get started.

%If you have a question, please use the support box in the bottom right of the screen to get in touch.
 
As was mentioned before in this document is reported the design and the implementation of an audio playback player over a stm32f4 discovery board, during the following sections we will see all the steps used for get this final result, in additional is recommended to the reader have a basic knowledge's on embedded systems, finite state machines and C language. 
\newline
\newline
Now the document structure is focus on the requirement analysis (Section 3), functional analysis (Section 4), architecture design (Section 5), Coding or implementation (Section 6), and finally a set of conclusion (section 7). 


   

\section{requirements}
This section contain the requirement specification at the user level which basically is a description of what the device must to do using natural language. Then we will move to the extraction of the functional and not functional requirements. 
\label{sec:Starting requirement}

\subsection{User Requirements}

Is necessary design a system able to reproduce a WAV audio format files, The data could be taken from a external disk as memory flash USB, or SDcard.  The system must to contain a visual element like LCD screen able to show the current state of the system and the elements inside of the disk in such way that the users are able to control the system and navigate around the folders, It means also that the user could chose a file inside of any folder for its reproduction. 
\newline
\newline
The Wav file have to contain the follow specifications:
\begin{enumerate}
\item Audio Format: PCM  uncompressed unsigned 8 bits.
\item Sample Rate :  8000,	11025, 22050, 44100 Hz
\item Bits Per Sample: 8-bit (range [0-255]).
\item Number of channels : monophonic.
\end{enumerate}
The system has to take the data store in the Wav file and verify the file specifications, and then start the reproduction. Others formats which are not supported as Mp3 or mpeg must to be indicated before to start the reproduction, In case in which a wrong format is selected or  the file don't satisfies the specification the system have to go back to the principal menu.
\newline
\newline
During the selection of the file the users have the possibility to navigate around the folders, thus is possible go forward or backward around them, and the reproduction must to start form any possible folder allocation. As addition, the system must to be portable, safety, and low power consumption.

\subsection{Functional Requirements}
Based in the user requirements, is possible obtain the functional requirements as follow:

\begin{enumerate}
\item Open/Read/Write/Close Operations on the SDCard.
\item Visualize the folders and files within of the SDCard.
\item Navigate  among the file system.
\item Validate The file before to reproduce it (parsing). 
\item Play/Stop the Wav Reproduction.
\end{enumerate}
Thus, the system could has certain physical characteristics which accomplish the above functional description which is reported as follow:

\begin{enumerate}
\item The system must to has a sdcard port able to interchange data as Input/Output.
\item The system must to Open/Close/Read/Write on the SDCard. 
\item The system must to has a internal memory such that make easy the data transfer. 
\item The system must to contain a screen in order to visualize the general state of the system.  
\item The system must to contain a DAC. "Digital to Analog convert" for the audio sound generation it could be trigger using a internal control signal.
\end{enumerate}

\subsection{Non Functional Requirements}
This implementation is a model which non required any specific production restriction, but required a shot time period for its development no more that 2 months of work, of course it must to be secure, easy to use, and has a special performance behavior, it is discoursed in the next section. In addition the implementation must to be monetarily economic it make easy its implementation and its production. 

\subsection{Performance Requirements}
As was mentioned before, is necessary implement a system able to express high performance behavior, it is doing  identified a set of operations: the visual part, the SDCard control operation, and finally the playback operation. Notes that this operations could be executed as a three different and independent tasks which cooperating through operations, Thus is important define a proper mechanisms for interchange information between each task. It guarantee very high performance during the execution of each task specially during the playback operation. 

\subsection{Input/Output Definition}


\subsubsection{•}

\subsection{Comments}

Comments can be added to the margins of the document using the \todo{Here's a comment in the margin!} todo command, as shown in the example on the right. You can also add inline comments too:

\todo[inline, color=green!40]{This is an inline comment.}

\subsection{Tables and Figures}

Use the table and tabular commands for basic tables --- see Table~\ref{tab:widgets}, for example. You can upload a figure (JPEG, PNG or PDF) using the files menu. To include it in your document, use the includegraphics command as in the code for Figure~\ref{fig:frog} below.

% Commands to include a figure:
%\begin{figure}
%\centering
%\includegraphics[width=0.5\textwidth]{SantanaPisa.png}
%\caption{\label{fig:frog}This is a figure caption.}
%\end{figure}

%\begin{table}
%\centering
%\begin{tabular}{l|r}
%Item & Quantity \\\hline
%Widgets & 42 \\
%Gadgets & 13
%\end{tabular}
%\caption{\label{tab:widgets}An example table.}
%\end{table}

\subsection{Mathematics}

\LaTeX{} is great at typesetting mathematics. Let $X_1, X_2, \ldots, X_n$ be a sequence of independent and identically distributed random variables with $\text{E}[X_i] = \mu$ and $\text{Var}[X_i] = \sigma^2 < \infty$, and let
$$S_n = \frac{X_1 + X_2 + \cdots + X_n}{n}
      = \frac{1}{n}\sum_{i}^{n} X_i$$
denote their mean. Then as $n$ approaches infinity, the random variables $\sqrt{n}(S_n - \mu)$ converge in distribution to a normal $\mathcal{N}(0, \sigma^2)$.

\subsection{Lists}

You can make lists with automatic numbering \dots

\begin{enumerate}
\item Like this,
\item and like this.
\end{enumerate}
\dots or bullet points \dots
\begin{itemize}
\item Like this,
\item and like this.
\end{itemize}

We hope you find write\LaTeX\ useful, and please let us know if you have any feedback using the help menu above.

\end{document}